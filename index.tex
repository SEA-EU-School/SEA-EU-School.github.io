% Options for packages loaded elsewhere
\PassOptionsToPackage{unicode}{hyperref}
\PassOptionsToPackage{hyphens}{url}
%
\documentclass[
]{article}
\usepackage{lmodern}
\usepackage{amssymb,amsmath}
\usepackage{ifxetex,ifluatex}
\ifnum 0\ifxetex 1\fi\ifluatex 1\fi=0 % if pdftex
  \usepackage[T1]{fontenc}
  \usepackage[utf8]{inputenc}
  \usepackage{textcomp} % provide euro and other symbols
\else % if luatex or xetex
  \usepackage{unicode-math}
  \defaultfontfeatures{Scale=MatchLowercase}
  \defaultfontfeatures[\rmfamily]{Ligatures=TeX,Scale=1}
\fi
% Use upquote if available, for straight quotes in verbatim environments
\IfFileExists{upquote.sty}{\usepackage{upquote}}{}
\IfFileExists{microtype.sty}{% use microtype if available
  \usepackage[]{microtype}
  \UseMicrotypeSet[protrusion]{basicmath} % disable protrusion for tt fonts
}{}
\makeatletter
\@ifundefined{KOMAClassName}{% if non-KOMA class
  \IfFileExists{parskip.sty}{%
    \usepackage{parskip}
  }{% else
    \setlength{\parindent}{0pt}
    \setlength{\parskip}{6pt plus 2pt minus 1pt}}
}{% if KOMA class
  \KOMAoptions{parskip=half}}
\makeatother
\usepackage{xcolor}
\IfFileExists{xurl.sty}{\usepackage{xurl}}{} % add URL line breaks if available
\IfFileExists{bookmark.sty}{\usepackage{bookmark}}{\usepackage{hyperref}}
\hypersetup{
  pdftitle={SEA-EU School on Computational Mathematics},
  hidelinks,
  pdfcreator={LaTeX via pandoc}}
\urlstyle{same} % disable monospaced font for URLs
\usepackage{graphicx}
\makeatletter
\def\maxwidth{\ifdim\Gin@nat@width>\linewidth\linewidth\else\Gin@nat@width\fi}
\def\maxheight{\ifdim\Gin@nat@height>\textheight\textheight\else\Gin@nat@height\fi}
\makeatother
% Scale images if necessary, so that they will not overflow the page
% margins by default, and it is still possible to overwrite the defaults
% using explicit options in \includegraphics[width, height, ...]{}
\setkeys{Gin}{width=\maxwidth,height=\maxheight,keepaspectratio}
% Set default figure placement to htbp
\makeatletter
\def\fps@figure{htbp}
\makeatother
\setlength{\emergencystretch}{3em} % prevent overfull lines
\providecommand{\tightlist}{%
  \setlength{\itemsep}{0pt}\setlength{\parskip}{0pt}}
\setcounter{secnumdepth}{-\maxdimen} % remove section numbering

\title{SEA-EU School on Computational Mathematics}
\author{}
\date{}

\begin{document}
\maketitle

\hypertarget{myNavbar}{}
\protect\hyperlink{home}{SEA-EU School on Computational Mathematics} {
\protect\hyperlink{overview}{{\emph{~}Overview}}
\protect\hyperlink{courses}{{\emph{~}Courses}}
\protect\hyperlink{lectures}{{\emph{~}Lectures}}
\protect\hyperlink{schedule}{{\emph{~}Schedule}}
\protect\hyperlink{registration}{{\emph{~}Registration}}
\protect\hyperlink{contact}{{\emph{~} Contact}} }
\href{javascript:void(0)}{\emph{}}

\href{javascript:void(0)}{Close ×}
\protect\hyperlink{overview}{Overview} \protect\hyperlink{work}{Courses}
\protect\hyperlink{team}{Lectures} \protect\hyperlink{pricing}{Scedule}
\protect\hyperlink{contact}{Contact}

\includegraphics[width=16.66667in,height=8.33333in]{img/Split_tower_view_ribbons.png}

\hypertarget{dive-into-mathematical-and-computational-modeling}{%
\paragraph{Dive into mathematical and computational
modeling}\label{dive-into-mathematical-and-computational-modeling}}

\emph{}

Oriented to Master, PhD students and young researchers in Mathematics,
Science or Engineering.

\emph{}

Erasmus Blended Intensive Program:\\

Sept. 2024:

2-6 Sept (online),\\
9-13 (Split, Croatia).

\emph{}

Prerequisites: Mathematics and computer programming skills.

\emph{}

3 courses \& 5 lectures in a friendly environment for enjoying
computational mathematics and applications.

\hypertarget{overview}{}
\includegraphics{img/Vestibule_julia.jpg}

\hypertarget{overview}{%
\subsubsection{Overview}\label{overview}}

This school is aimed at young students looking to acquire skills on the
\textbf{most trending techniques in mathematical and computational
modeling}. We place a strong emphasis in practical applications in
science and engineering, with a focus on modeling spatiotemporal
variations of physical quantities using partial differential equations
(PDEs). And related topics such as numerical optimization, playing a
vital role today in fields like deep learning.

The school is structured into \textbf{three courses}. Two of them cover
the widely used finite element method (FEM) and other emerging
techniques such as physically informed neural networks (PINN) for
computing numerical solutions of PDEs. A third course focus on numerical
methods for constrained and unconstrained optimization. In these
courses, we introduce fundamental math structures, provide illustrative
examples, review algorithms, program them with high-level open-source
tools, and post-process the solutions for high-quality plotting.

The necessary mathematical and computer skills will be standardized in
introductory online sessions, with the entire learning process overseen
by expert professors in the field. The courses are complemented by
\textbf{five conferences} where lecturers related to SEA-EU universities
introduce recent research projects related to the subjects of the
School.

Methodology

The school begins with a phase dedicated to homogenizing the students'
knowledge to help them reach a basic level before the classroom courses
begin to develop in Split. It will be designed as \textbf{5 online days}
where students will be able to access material specifically prepared for
them to acquire basic math and computer skills. During these days they
will have the support and tutoring of the curses instructors, through
video meetings and forums in a virtual campus.

In a \textbf{second week}, the School will be held physically on the
premises of the \textbf{University of Split}. It will consist on three
courses where theoretical lectures will be given by the professor and
numerous examples, algorithms and exercises will be reviewed. In
computer classes, students will be provided with the necessary software
to implement them and analyze the results. The courses will be
complemented with conferences where invited professors will stimulate
the curiosity of the students, showing recent research topics related
with the matter.

After the end of the classroom days in Split, the students interested in
evaluation to be recognized up to three credits, will deliver a final
work in small groups related to the school topics. For this, they will
be assisted by a final tutoring session.

Credit Recognition

Some students might be interested to be recognized for up to three
credits. For that we have foreseen an evaluation process, which will be
based on daily attendance to the school and a small group work. It will
be related to the school topics and will be delivered after the end of
the classroom sessions.

\includegraphics{img/dormitory-blowup.jpg}

Travel and Accommodation

The city of Split, Croatia, is served by an international airport which
is located approximately 20 km from the city center of Split, on the
west side of Kaštela Bay. Regular taxis at Split Airport are available
during the airport operation times, being the average cab fare to the
center of Split about 30€. Other options for travelling include Shuttle
Bus (Pleso Prijevoz) and Local Bus, with respective costs of 6€. and
2.5€.

The School will provide students with up to 20 places in the university
residence located on the university campus, a 5-minute walk from the
faculty of science, where the activities will take place. Breakfast and
lunch are included.

Erasmus Funding

Up to 20 Students coming from a EU Member State or third country
associated to the Erasmus+ program can apply for funding through Blended
Intensive Program (BIP). This will allow them to get support to attend
the classroom week in Split.

\protect\hyperlink{work}{\emph{~}Registration}

\hypertarget{courses}{%
\subsection{Courses}\label{courses}}

\textbf{Course 1}. Introduction to partial differential equations:
examples and numerical resolution by the finite element method

\emph{Organizer \& Speaker:} \textbf{Francisco Ortegón Gallego}
(Universidad de Cádiz)

\emph{Classroom time:} \textbf{} 10 hours

\emph{Topics:} \textbf{}

\begin{enumerate}
\tightlist
\item
  Basic notions: differential operators and integral identities.
\item
  Some PDEs arising in physics and engineering.
\item
  Variational formulation. Functional spaces. The finite element method.
\item
  Introduction fo \texttt{Freefem++}: numerical resolution of PDEs.
\item
  Working on a 3D problem: 3D tetrahedralization, resolution and
  post-processing.
\end{enumerate}

\includegraphics{img/palace_atrio.jpg}

\begin{center}\rule{0.5\linewidth}{0.5pt}\end{center}

\textbf{Course 2}. Physics-informed neural networks: introduction and
case study on fluid dynamics

\emph{Organizers \& Speakers:} \textbf{J. Rafael Rodríguez Galván, M.
Victoria Redondo Neble} (Universidad de Cádiz)

\emph{Classroom time:} \textbf{} 10 hours

\emph{Topics:} \textbf{}

\begin{enumerate}
\tightlist
\item
  Introduction and mathematical foundations of neural networks (NN).
  Fundamentals, architectures, activation and loss functions,
  differentiation and chain rule for functions of several variables
\item
  Training and optimization in NN. Backpropagation, optimization
  algorithms
\item
  Computational aspects and software libraries. Significance of hardware
  acceleration and parallelization for efficient training
\item
  A perspective on neural networks for PDE models: physics informed
  neural networks (PINN). PINN software libraries. Modeling diffusion
  and convection for linear and non-linear processes
\item
  Governing equations in fluid dynamics. The finite element method (FEM)
  for approximating non-turbulent flows
\item
  PINN in fluid dynamics: comparing to FEM, exploring the pros and cons
  in a challenging case with applications to real-world scenarios
\end{enumerate}

\begin{center}\rule{0.5\linewidth}{0.5pt}\end{center}

\textbf{Course 3}. Numerical optimization

\emph{Organizer \& Speaker:} \textbf{Malte Braack} (University of Kiel)

\emph{Classroom time:} \textbf{} 6 hours lectures, 4 hours exercises

\emph{Topics:} \textbf{}

\begin{enumerate}
\tightlist
\item
  Numerical methods for unrestricted optimization: Newton, steepest
  decent, Armijo step length control
\item
  Restricted Optimization Problems: equality and inequality constraints,
  types of restricted optimization problems
\item
  Sequential Unrestricted Minimization Technique (SUMT): penalty method,
  SUMT for equality constraints, SUMT for inequality constraints
\item
  Stationary points for Restricted Optimization Problems: first-order
  necessary condition, active sets, linearized tangential cones, Abadie
  constraint qualification, Lagrange function, Karush-Kuhn-Tucker (KKT)
  system, Farkas lemma
\item
  Convex optimization and Slater condition: convex constraints, relation
  between Slater condition and KKT
\item
  Numerical methods based on Karush-Kuhn-Tucker system: Lagrange-Newton
  for restricted optimization, Sequential Quadratic Programing (SQP)
\end{enumerate}

\hypertarget{lectures}{}
\includegraphics{img/Sphinx_Dioclecian.jpg}

\hypertarget{lectures}{%
\subsection{Lectures}\label{lectures}}

\textbf{Lecture 1}. Saša Krešić-Jurić (University of Split, Croatia)

\emph{Introduction to Symplectic Methods for Hamiltonian Systems}

Hamilton's equations describe the time evolution of a mechanical system
defined on a symplectic manifold. Such systems have a wide range of
applications from celestial mechanics, rigid body motion to molecular
dynamics.

When discretizing Hamilton's equations, one wishes to preserve the
geometric properties important for qualitative behavior of the system.
In this talk we give a short introduction to symplectic integrators for
Hamiltonian systems based on preservation of the symplectic form under
the flow of a Hamiltonian vector field. We discuss the symplectic Euler
and Stromer-Verlet algorithms as well as composition methods for
symplectic maps. We show that these algorithms outperform standard
methods in terms of accuracy and stability for long time integrations.
These methods are illustrated with several interesting examples from
physics.

\begin{center}\rule{0.5\linewidth}{0.5pt}\end{center}

\textbf{Lecture 2}. Hermenegildo Borges de Oliveira (University of
Algarve, Portugal)

\emph{Theoretical and numerical analysis of Navier-Stokes equations
arising in fluid confinement}

In this lecture, we consider the incompressible Navier-Stokes equations
with the forcing term assumed to be (non-linearly) dependent on the
velocity field. Some applications of this problem will be shown. For the
considered problem, we characterize feedback forces fields that are able
of confining the fluid flow. We use the Continuous/Discontinuous Finite
Element Method with interior penalty terms to solve the resulting
nonlinear fourth-order problem. For the associated continuous and
discrete problems, we prove the existence of weak solutions and
establish the conditions for their uniqueness. Consistency, stability,
and convergence of the numerical method are also shown analytically. To
validate the numerical model regarding its applicability and robustness,
several test cases are carried out.

\begin{center}\rule{0.5\linewidth}{0.5pt}\end{center}

\textbf{Lecture 3}. Karolina Kropielnicka (IM PAS, Poland)

\emph{Split in Split}

Splitting methods are well established and widely used techniques for
finding approximate solutions of linear DEs of the form
\emph{u'=(A+B)u}. They can be also used for the case of time dependent
component \emph{B(t)}.

In this lecture I will give a short introduction to this subject on the
example of celebrated Strang splitting for possibly time dependant
component. I will show, that surpassingly it can be derived from Duhamel
(Variation-of-Constant) formula. Based on this approach I will present a
new proof of convergence of this scheme and elaborate on the
possibilities brought by this approach.

\begin{center}\rule{0.5\linewidth}{0.5pt}\end{center}

\textbf{Lecture 4}. Andrijana Curkovic (University of Split, Croatia)

\emph{Introduction to Asymptotic Methods in Fluid Mechanics}

The application of asymptotic analysis to partial differential equations
is presented using the example of a fluid flow through a thin pipe that
is heated. The expansion of the pipe is considered and the small system
parameters are the expansion coefficient of the pipe and the radius of
the circular pipe.

\begin{center}\rule{0.5\linewidth}{0.5pt}\end{center}

\textbf{Lecture 5}. M. Conepción Muriel Patino (University of Cádiz,
Spain)

\emph{Solvable Structures and C-infinity Structures for Differential
Equations}

We present some new integration methods for ordinary differential
equations, based on the existence of
\includegraphics{img/Conchi_html_989b7876ff8a151.gif}-structures, a
recent generalization of the concept of solvable structure. Both notions
are established in the more general context of the integrability of
involutive distributions \emph{Z} of vector fields.

Several illustrative examples show how both objects (solvable structures
and \includegraphics{img/Conchi_html_989b7876ff8a151.gif}-structures)
can be found and used in practice to obtain exact solutions for problems
modelled by ordinary differential equations.

\begin{center}\rule{0.5\linewidth}{0.5pt}\end{center}

\hypertarget{team}{}
\hypertarget{lecturers-and-organizers}{%
\subsection{Lecturers and organizers}\label{lecturers-and-organizers}}

\hfill\break

% \includegraphics{https://produccioncientifica.uca.es/img/uploaded/DB772852C75DDC61A7DD0916CB7C1792.jpg}

Francisco Ortegón Gallego

Universidad de Cádiz, Spain

% \includegraphics{https://produccioncientifica.uca.es/img/uploaded/A1520884E59249A40BDA0799956F143E.jpg}

J. Rafael Rodríguez Galván

Universidad de Cádiz, Spain

% \includegraphics{https://produccioncientifica.uca.es/img/uploaded/15235392317FA096335E3C2B4CB66126.jpg}

M. Victoria Redondo Neble

Universidad de Cádiz, Spain

% \includegraphics{https://media.githubusercontent.com/media/SEA-EU-School/SEA-EU-School.github.io/main/img/MBraack.jpeg}

Malte Braack

Kiel University, Germany

\hfill\break

% \includegraphics{https://mapmf.pmfst.unist.hr/~skresic/Sasa_2.jpg}

Saša Krešić-Jurić

University of Split, Croatia

% \includegraphics{https://media.githubusercontent.com/media/SEA-EU-School/SEA-EU-School.github.io/main/img/HGildo.png}

Hermenegildo Borges de Oliveira

Universidade do Algarve and CMAFcIO, Portugal

% \includegraphics{https://mat.ug.edu.pl/~kmalina/karo/karolina11.png}

Karolina Kropielnicka

Institute of Mathematics of the Polish Academy of Sciences

% \includegraphics{https://produccioncientifica.uca.es/img/uploaded/807E73F92075E45A662CEC34CDF40F09.jpg}

Maria Concepcion Muriel Patino

Universidad de Cádiz, Spain

\hfill\break

~

% \includegraphics{https://media.githubusercontent.com/media/SEA-EU-School/SEA-EU-School.github.io/main/img/Andrijana.jpg}

Andrijana Curkovic

University of Split, Croatia

% \includegraphics{https://scholar.google.de/citations/images/avatar_scholar_128.png}

Marcin Marciniak

Uniwersytet Gdańsk, Poland

{=~5}\\
Universities from the SEA-EU alliance

{=~4~+~5}\\
Speakers in courses + lectures

{=~5~+~5}\\
Online + face-to-face days

{≥~20}\\
Tutored hours of computer practice exercises

\hypertarget{schedule}{%
\subsection{Schedule}\label{schedule}}

\hypertarget{schedule-for-online-introduction}{%
\subparagraph{Online Introduction Week (September 2-6
2024)}\label{schedule-for-online-introduction}}

\begin{itemize}
\item
  \textbf{Monday 2}, 9:00h. Opening (video conference). Introduction to
  Course 1 (video conference).\\
  10:00h \emph{Francisco Ortegón Gallego}. Individual work, tutored in
  forums on the virtual campus
\item
  \textbf{Tuesday 3}, 9:00h. Introduction to Course 2 (video
  conference). \emph{J. Rafael Rodríguez Galván \& M. Victoria Redondo
  Neble}. Individual work, tutored in forums on the virtual campus
\item
  \textbf{Wednesday 4}, 9:00h. Introduction to Course 3 (video
  conference). \emph{Malte Braack}. Individual work, tutored in forums
  on the virtual campus
\item
  \textbf{Thursday 5} and \textbf{Wednesday 6}. Guided exercises.
\end{itemize}

\hypertarget{schedule-for-classroom-sessions}{%
\subparagraph{Schedule for Classroom Sessions (September 9-13
2024)}\label{schedule-for-classroom-sessions}}

\includegraphics{img/schedule.png}

\hypertarget{schedule-for-evaluation}{%
\subparagraph{Schedule for Evaluation}\label{schedule-for-evaluation}}

Those students interested in evaluation, for credits recognition, must
submit a final project before Friday, 20th September, 2024. may be done
individually or in small groups.

\hypertarget{modal01}{}
{×}

\hypertarget{topics}{%
\subsection{Topics}\label{topics}}

An insight of mathematical and computational tools for simulation of
models from Science and Engineering, specifically those which are
written as PDE.

With a first course on the finite element method (FEM) for numerical
approximation of PDE, a second one on Artificial Intelligence for
learning solutions to PDE, in particular by delving into physically
informed neural networks (PINN), and a third course on numerical methods
for constrained and unconstrained optimization.

\emph{}Computer Simulation of PDE Models

70\%

\emph{}Artifical Intelligence

33\%

\emph{}Numerical optimization

33\%

\hypertarget{registration}{%
\subsection{Registration}\label{registration}}

Registration for the school will be activated shortly.

\hypertarget{contact}{%
\subsection{Contact}\label{contact}}

\hypertarget{to-do-design-this-section}{%
\subparagraph{\texorpdfstring{\textbf{TO-DO: design this
section}}{TO-DO: design this section}}\label{to-do-design-this-section}}

\emph{} Split, Croatia

\emph{} Phone: +00 151515

\emph{} Email: mail@mail.com

\hfill\break

\emph{}~Send Message

\includegraphics{img/Split_view.jpg}

\emph{See you in Split!}

\protect\hyperlink{home}{\emph{}To the top}

\emph{} \emph{} \emph{} \emph{}

Powered by \href{https://www.w3schools.com/w3css/default.asp}{w3.css}
All the images are licensed under
\href{https://commons.wikimedia.org/wiki/Category:CC-BY-SA-4.0}{CC-BY-SA-4}
license. All of them are © J. Rafael Rodríguez Galván exepting
\emph{Vestibule} and \emph{Sphinx} in \emph{Diocletian's Palace}, which
are derived from images by Goran Leš and Adam Jones
(\href{https://commons.wikimedia.org}{Wikimedia}).

\end{document}
